%%%
%%% Abstract.
%%%

\svnInfo $Id$


\chapter*{\pdfbookmark[0]{Zusammenfassung}{chapter*.1}Zusammenfassung\revision}
\addtocontents{toc}{\protect \contentsline {chapter}{Zusammenfassung\revision}{\thepage }{}}


Im Bereich der multi-objective Optimierung ist es aus Anwendersicht
manchmal schwierig geeignete Zielfunktion zu finden.
%
Der \emph{Skyline Operator} filter aus einer potentiell gro\ss{}en
Datenmenge \emph{interessante} Tuple heraus.
%
Ein Tuple geh\"ort genau dann zur \emph{Skyline}, wenn es nicht durch
anderes \emph{dominiert} wird, d.h. es gibt kein Tuple das in allen
Kriterien zumindest gleich gut ist und in zumindest einem besser ist.
%
Unabh\"angig wie die Pr\"aferenzen innerhalb der Attribute gew\"ahlt
werden, sind nur jene Tuple, die unter einer \emph{monotonen
Scoring-Funktion} am besten bewertet werden, Teil der Skyline.
%
Mit anderen Worten, die Skyline schlie\ss{}t alle Tuple aus die
niemand als Favorit hat.
%
Das Konzept der Skyline ist auch unter dem Namen \emph{Pareto
Optimalit\"at} und die ihre Berechnung als \emph{Maximum Vector
Problem} bekannt.


Diese Diplomarbeit pr\"asentiert unsere Arbeit PostgreSQL, ein Open
Source relationales Datenbanksystem, um den Skyline Operator zu
erweitern.
%
Unsere Implementierung bietet verschiedene \emph{physische Operatoren}
um die Skyline zu berechnen und zwar: BNL, SFS und eine Variante von
LESS.
%
Mit dieser Arbeit hoffen wir den Boden zu ebnen f\"ur weitere
Forschung und den Skyline Operator von standalone Implementierungen in
sein nat\"urliches Habitat, das RDBMS, zu verpflanzen.
%
Des Weiteren haben wir einen Fehler in der Originalversion von BNL
entdeckt und liefern dazu eine korrigierte Version.


Es ist wohlbekannt, dass die Performance von Skyline Abfragen von einer
Reihe von Parametern abh\"angt.
%
Aus umfangreichen Experimenten mit unserer Implementierung haben wir
mehrere Regeln abgeleitet, die bemerkenswert einfach und n\"utzlich
sind, aber nur schwer durch theoretische Untersuchungen zu gewinnen
sind.
%
Unsere Resultate sind von Vorteil bei der Entwicklung von Heuristiken
f\"ur Skyline Abfrage-Optimierung und in der Zwischenzeit liefern sie
einen Beitrag zum tieferen Verst\"andnis der Skyline Abfrage
Charakteristik.


Alle Resultate und der Source-Code sind auf
\url{http://skyline.dbai.tuwien.ac.at/} online verf\"ugbar.
