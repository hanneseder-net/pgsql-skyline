%%%
%%% Abstract.
%%%

\svnInfo $Id$


\chapter*{\pdfbookmark[0]{Zusammenfassung}{chapter*.1}Zusammenfassung\revision}
\addtocontents{toc}{\protect \contentsline {chapter}{Zusammenfassung\revision}{\thepage }{}}


% multi-objective <=> multikriteriell
% http://dict.leo.org/forum/viewUnsolvedquery.php?idThread=532247
Im Bereich der multikriteriellen Optimierung ist das Design einer
geeigneten Zielfunktion aus Anwendersicht eine Herausforderung.
%
Der \emph{Skyline Operator} filtert aus einer potentiell gro\ss{}en
Datenmenge \emph{interessante} Tupel heraus.
%
Ein Tupel geh\"ort genau dann zur \emph{Skyline}, wenn es nicht durch
ein anderes \emph{dominiert} wird, d.h. es gibt kein Tupel, das in allen
Kriterien zumindest gleich gut ist und in zumindest einem besser ist.
%
Unabh\"angig wie die Pr\"aferenzen innerhalb der Attribute gew\"ahlt
werden, sind nur jene Tupel, die unter einer \emph{monotonen
Scoring-Funktion} am besten bewertet werden, Teil der Skyline.
%
Mit anderen Worten, die Skyline schlie\ss{}t alle Tupel aus, die
niemand als Favorit hat.
%
Das Konzept der Skyline ist auch unter dem Namen \emph{Pareto
Optimalit\"at} und ihre Berechnung als \emph{Maximum Vektor
Problem} bekannt.


Ziel dieser Diplomarbeit ist PostgreSQL, ein Open Source relationales
Datenbankmanagementsystem (RDBMS), um den Skyline Operator zu
erweitern und Skyline Algorithmen im RDBMS Kontext zu evaluieren.
%
Das Endziel ist eine Skyline-Abfrage-Optimierung zu entwickeln, 
die automatisch gute Abfragepl\"ane bez\"uglich I/O, Zeit und
Speicherverbrauch erstellt.
%
Diese Arbeit ebnet den Boden f\"ur weitere Forschung und verpflanzt
den Skyline Operator von standalone Implementierungen in sein
nat\"urliches Habitat, das RDBMS.


Unsere Implementierung bietet verschiedene \emph{physische
Operatoren}, um die Skyline zu berechnen, allen voran: BNL, SFS und
eine Variante von LESS.
%
Zus\"atzlich erweitern wir die Standardsyntax, um Semantik und
verschiedene operationale Aspekte zu beeinflussen.
%
Als Nebenresultat unserer Arbeit haben wir einen Fehler in der
Originalversion von BNL entdeckt und liefern dazu eine korrigierte
Version.
%
Wir schlagen ein neues Einsatzgebiet f\"ur den Elimination Filter (EF)
vor und zwar: BNL+EF. Diese Kombination stellt eine erhebliche
Verbesserung gegen\"uber BNL dar.

Es ist eine bekannte Tatsache, dass die Performance von Skyline
Abfragen von einer Reihe von Parametern abh\"angt.
%
Aus umfangreichen Experimenten mit unserer Implementierung haben wir
mehrere bemerkenswert einfache und n\"utzliche Regeln abgeleitet, die
nur schwer theoretisch zu gewinnen sind.
%
Unsere Resultate helfen Heuristiken f\"ur die Skyline-%
Abfrage-Optimierung zu entwickeln und liefern einen Beitrag zum
tieferen Verst\"andnis der Skyline-Abfrage-Charakteristik.


Alle Resultate, der Source-Code und ein Web-Interface zum Testen
unserer Implementierung sind auf
\url{http://skyline.dbai.tuwien.ac.at/} verf\"ugbar.
