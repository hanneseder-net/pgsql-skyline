%%%
%%% abstract.
%%%

\svnInfo $Id$


\chapter*{\pdfbookmark[0]{Abstract}{chapter*.2}Abstract\revision}
\addtocontents{toc}{\protect \contentsline {chapter}{Abstract\revision}{\thepage }{}}


In the realm of multi-criteria optimization designing an appropriate
objective function is a challenging task from a user's perspective.
%
The \emph{skyline operator} filters out the \emph{interesting} tuples
from a potentially large dataset.  A tuple belongs to the \emph{skyline} if
it is not \emph{dominated} by any other tuple, i.e. there is no tuple
which is at least as good as in all and better in at least one criteria.
%
No matter how we weight our preferences along the attributes, only
those tuples which score best under a \emph{monotone scoring function} are
part of the skyline.
%
In other words, the skyline does not contain tuples which
are nobody's favorite.
% The skyline exclues exactly the tuples which are nobody's favorite
%
The notion of skyline is also called \emph{Pareto optimal set} and its
computation \emph{maximum vector problem}.


In this thesis we aim at extending PostgreSQL, an open source
relational database management system (RDBMS), with the skyline
operator and the evaluation of skyline algorithms in the RDBMS
context, with the ultimate goal of building a skyline query optimizer
to automatically generate a good query plan w.r.t. I/O, time, and
memory consumption.
%
This effort lays the ground for future work in this area and moves the
skyline operator from standalone implementations to the habitat it
belongs to: an RDBMS.


Our implementation provides several physical operators for computing
the skyline, including: BNL, SFS, and a variant of LESS.
%
In addition, we extend the standard syntax to influence the
semantics and various operational aspects.
%
As a byproduct of our work, we discovered a flaw in the original
version of BNL and give a corrected version.
%
We propose a new use case for the elimination filter (EF), namely:
BNL+EF. It turns out that BNL+EF is a substantial improvement to BNL.

It is well known that the performance of skyline queries is sensitive
to a number of parameters.  
%
Extensive experiments on skyline implementations helped us to discover
several remarkably simple and useful rules, which are hard to obtain
from theoretical investigations.
%
Our findings are beneficial for developing heuristics for the skyline
query optimization, and in the meantime, provide some insight for a
deeper understanding of the skyline query characteristics.

All results, the source code, and a web-interface to test-drive the
implementation are available at:
\url{http://skyline.dbai.tuwien.ac.at/}.
