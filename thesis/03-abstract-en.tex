%%%
%%% abstract.
%%%

\svnInfo $Id$


\chapter*{Abstract\revision}
\addcontentsline{toc}{chapter}{Abstract\revision}


In the realm of multi-criteria optimization it is sometimes hard from
a user's perspective to come up with an appropriate objective
function.
%
The \emph{skyline operator} filters out the \emph{interesting} tuples
from a potentially large dataset.  A tuple belongs to the skyline if
it is not \emph{dominated} by any other tuple, i.e. there is no tuple
which is at least as good in all criteria and better in at least one.
%
No matter how we weight our preferences along the attributes, only
those tuples which score best under a monotone scoring function are
part of the skyline.
%
To state it in other words, the skyline does not contain tuples which
are nobody's favorite.
%
The notion of skyline is also called \emph{Pareto optimal set} and the
computation of it \emph{maximum vector problem}.


In this thesis we present our work on extending PostgreSQL, an open
source relational database system, with the skyline operator.
%
Our implementation provides several physical operators for computing
the skyline, namely: BNL, SFS, and a variant of LESS.
%
With this effort we hope to lay the ground for future work in this
area and move the skyline operator from standalone implementations to
the habitat it belongs to: an RDBMS.
%
Moreover we discovered a flaw in the original version of BNL and give
a corrected version.

It is well known that the performance of skyline queries is sensitive
to a number of parameters.  
%
From extensive experiments on skyline implementations we have
discovered several rules, which are remarkably simple and useful, but
hard to obtain from theoretical investigation.
%
Our findings are beneficial for developing heuristics for the skyline
query optimization, and in the meantime, provide some insight for a
deeper understanding of the skyline query characteristics.

All results and source code are available at:
\url{http://skyline.dbai.tuwien.ac.at/}.
