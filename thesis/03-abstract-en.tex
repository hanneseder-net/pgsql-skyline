%%%
%%% abstract.
%%%

\svnInfo $Id$


\chapter*{Abstract\revision}
\addcontentsline{toc}{chapter}{Abstract\revision}

The \emph{skyline operator} filters out the \emph{interesting} tuples
from a potentially large dataset.  A tuple is of interest if not
dominated by any other tuple.  The skyline is also called 
\emph{pareto optimal set} and the computation of it as
\emph{maximum vector problem}.

In this paper, we present our work on extending PostgreSQL, an open
source database system, with the skyline operator.  Our implementation
provides several physical operators for computing the skyline, namely:
BNL, SFS, and a variant of LESS.  While at it we discovered and fixed
a flaw in the original version of BNL.  With this effort we hope to
lay the ground for future work in this area and move the skyline
operator from standalone implementations to the habitat it belongs to:
an RDBMS.

It is well known that the performance of skyline queries is sensitive
to a number of parameters.  From extensive experiments on skyline
implementations we have discovered several rules, which are remarkably
simple and useful, but hard to obtain from theoretical investigation.
Our findings are beneficial for developing heuristics for the skyline
query optimization, and in the meantime, provide some insight for a
deeper understanding of the skyline query characteristics. 

All results and source code is available at: 
\url{http://skyline.dbai.tuwien.ac.at/}.
