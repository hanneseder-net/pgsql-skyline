%%%
%%% Introduction.
%%%

\svnInfo $Id$

\newcommand\schema[1]{\ensuremath{\mathcal{#1}}}
\newcommand\relation[1]{\ensuremath{\textnormal{\bf\textsf{#1}}}}

\newcommand\imp{\Rightarrow}
\newcommand\bigland{\bigwedge}
\newcommand\biglor{\bigvee}

\newcommand\inlinesql[1]{{\tt #1}}
\newcommand\srcref[1]{{\tt #1}}
\newcommand\postgresdocu[2]{\href{http://www.postgresql.org/docs/8.3/static/#1}{#2}}

\newtheorem{definition}{Definition}
\newtheorem{lemma}{Lemma}
\newtheorem{theorem}{Theorem}

\chapter{Introduction\revision}
\label{chap:Introduction}


\subsection{motivation}
motivation: green computing: less cpu time, less computers, less energy --> safe the planet

stocks: risk vs. costs (--> streaming skyline see [Tao2006])

example of skyline

when travelling by plan: time vs. costs

TODO: more dim: \#stops, total dist
travel by bus: \#of visited cities

it is shown on some website. which?

Credit To: Ali (room mate)

employee (salary, revenue, years in the company)

\subsection{consider only bnl and sfs}

In this paper, our focus is on algorithms that can be
directly implemented in today's commercial database systems
without the addition of new access methods (which
would require addressing the associated challenges of maintenance
with updates, concurrency control, etc.). Specifically,
we consider the Block-nested-loop Algorithm [15],
and the Block-nested-loop with Presorting [5]. There are
other algorithms that include Bitmap and Index[16], NN
[7] and BBS [12] which use specialized data structures like
extensions of B-trees and R$^*$-trees. We leave the cost analysis
of these algorithms and indexes for future work.


\todo{}{take some parts from the research planning intro}
The skyline operator \citep{Borzsonyi2001} filters out the \emph{interesting} points from a potentially large set of data points. The skyline operator returns the pareto optimal\index{pareto optimal} elements of a set. This is also know as maximum vector problem\index{maximum vector problem}.

Although we continously merging \texttt{CSV HEAD} of the \fixme{PostgreSQL source repository}{ref} 
into our main development branch, we decided to base this paper on branch \texttt{REL8\_3\_0}, 
in order to have a more stable and reproducibly setting, i.e. all experiments in this paper are based on version 8.3.0 of PostgreSQL with our patch for the skyline operator applied.

\section{Related Problems}
Convex hull\index{convex hull} is a related problem.



\newcommand\dominates{\ensuremath{\succ}\xspace}

\clearpage 
\fixme{}{no clearpage}
\section{Basic definitions}
\todo{We are in the context of relational model\index{relational model} of data}{\citep{Chomicki2002, Chomicki2003a}?}. As we aim at an implementation of the skyline operator into a relational database management system (RDBMS)\index{RDBMS} we restrict to finite database\index{finite database} instances.

We assume two infinite domains: $N$ (numbers) and $D$ (uninterpreted constants). The domain $D$ will be used for attributes which are not subject to skyline computation, i.e. for the travelling example we will not compute the skyline over the name of the airline, therefore $D$ will be used as domain for this attribute.
For attributes or expressions which are subject to skyline computation we use the domain $N$. No distinction is made between different numeric domains, since it is not necessary for this paper. For $N$ we require that equality ($=$), inequality ($\not=$), and the binary relations $<$ (strictly less than), $\le$ (less than or equal), $\ge$ (greater than or equal), and $>$ (strictly greater than) are defined with the usual properties.

\begin{definition}[Schema]
Given the domains $U_i, 1 \le i \le n$, such that $U_i$ is either equal to $D$ or $N$, we define the $n$-ary schema $\schema{R}$ as the cartesian product of the $U_i$'s, i.e. $\schema{R} = U_1 \times U_2 \times \ldots \times U_n$. The attributes of schema \schema{R} will be refered as $a_1, \ldots, a_n$.
\end{definition}

Preferences will be defined in terms of \emph{binary preference relations}\index{preference relation}.
\begin{definition}[Preference relation]
Let \schema{R} be a $n$-ary schema, a relation \dominates is a preference relation over \schema{R} if it is a subset of $\schema{R} \times \schema{R}$.
\end{definition}

To give an intuition, \dominates will be a binary relation between pairs of tuples from the same (database) relation, we say $r$ \emph{dominates} $s$ in \dominates iff $(r, s) \in \dominates$ (or in infix notation $r \dominates s$).

We require the relation \dominates to be a \emph{strict partial order}\index{strict partial order}, i.e. \dominates is irreflexive, asymmetric and transitive. These properties are formalized as usual:

\begin{itemize}
\item \emph{irreflexivity:} $\forall x: x \not\dominates x$
\item \emph{asymmetry:} $\forall x, y: x \dominates y \imp y \not\dominates x$
\item \emph{transitivity:} $\forall x, y, z: (x \dominates y \land y \dominates z) \imp x \dominates z$
\end{itemize}

Non-transitive\index{non-transitive} preferences can be compute with the \todo{Best}{special font for algo names?} algorithm (\citep{Torlone2002, Ciaccia2004}), we will not study this approach in this paper.

\begin{definition}
\todo{}{}Let \relation{R} be a relation of schema \schema{R}.
\end{definition}

\begin{definition}[Skyline]
The skyline of a relation \relation{R} with respect to the preference relation \dominates is the set of tuples $r \in \relation{R}$ which are not dominate by any other tuple $s \in \relation{R}$, formally $\skyline_\dominates(\relation{R}) := \{ r \in \relation{R} | \nexists s \in \relation{R} : s \dominates r \}$
\end{definition}

\begin{lemma}
\todo{}{this is more or less theorem 2 from \citep{Chomicki2002}}
Given the relation $\relation{R} \not= \emptyset$ and \dominates a strict parial order over \schema{R}, where \schema{R} is the schema for \relation{R}, the skyline is non empty, i.e. $\skyline_\dominates(\relation{R}) \not= \emptyset$ holds.
\end{lemma}


In a skyline query with the following skyline clause:
\[
\texttt{SKYLINE OF} a_1 \texttt{MIN}, \ldots, a_k \texttt{MIN}, a_{k+1} \texttt{MAX}, \ldots, a_l \texttt{MAX}, a_{l+1} \texttt{DIFF}, \ldots, a_m \texttt{DIFF}
\]
a tuple 
$r = (r_1, \ldots, r_k, r_{k+1}, \ldots, r_l, r_{l+1}, \ldots, r_m, r_{m+1}, \ldots, r_n)$
dominates a tuple
$s = (r_1, \ldots, r_k, r_{k+1}, \ldots, r_l, r_{l+1}, \ldots, r_m, r_{m+1}, \ldots, r_n)$
iff the following condition holds
\begin{equation}\label{equ:skylinepf}
\left( \bigland_{1 \le i \le k} r_i \le s_i \right) \land
\left( \bigland_{k+1 \le i \le l} r_i \ge s_i \right) \land
\left( \bigland_{l+1 \le i \le m} r_i = s_i \right) \\
\land
\left( \left( \biglor_{1 \le i \le k} r_i < s_i \right) \lor
       \left( \biglor_{k+1 \le i \le l} r_i > s_i \right) \right).
\end{equation}

Please note that the following subformula of the upper formula induces an equality relation on \relation{R}, i.e. \relation{R} is partitioned into groups where the attributes $a_{l+1}, \ldots, a_m$ are equal:
\[
\bigland_{l+1 \le i \le m} r_i = s_i.
\]

As noted in \citep{Chomicki2003}, the \inlinesql{DIFF} directive works as a group by within the skyline, and the skyline for each group of diff attributes' values is found.

This property can be exploited, if an ordered index on any subset of the attributes $a_{l+1}, \ldots, a_m$ exists, since the tuple window can be flushed each time the group is changed.

%\subsection{Skyline as special case of winnow}
The skyline operator is a special case of the \emph{winnow operator}\index{winnow operator} \citep{Chomicki2002}, where the preference formula has exactly the form of formula (\ref{equ:skylinepf}).

Our implementation of the \inlinesql{SKYLINE OF} clause is actually a bit more flexible than the mathematical definition given above. With our implementation the skyline operator is not restricted to attributes with a numerical domain, it can be applied to any attribute, as long as a \emph{sort function} is defined for the domain in question, i.e.\/ any expression valid in a SQL \inlinesql{ORDER BY} clause is valid as an expression in a \inlinesql{SKYLINE OF} clause.

This gives the opportunity to include expressions of almost any data type in a skyline query, even user defined ones. For more information on user defined data types see PostgreSQL documentation on \postgresdocu{xtypes.html}{User-Defined Types} and \postgresdocu{xindex.html\#XINDEX-OPFAMILY}{Operator Classes and Operator Families}

Anyway it is somewhat questionable what it is good for the include a e.g. \inlinesql{VARCHAR} column in a skyline query, still it is possible.

Furthermore our implementation allows arbitrary expressions instead of a single attribute.


\subsection{Monoton vs. linear scoring functions}
see \citep{Chomicki2002a} page 5, Theorem 4

\section{Preliminaris}
\section{Existing Methods and Related Works}
\section{Skyline Algorithms}
\subsection{BNL}
\subsection{SFS}
\subsection{LESS}

\chapter{Implementation}
\section{SQL Extension}

In \citep{Borzsonyi2001} the following extension to the SQL's \inlinesql{SELECT} statement with an optional \inlinesql{SKYLINE OF} clause is proposed:
\begin{sql}
SELECT ... FROM ... WHERE ... \\
GROUP BY ... HAVING ...       \\
SKYLINE OF \textnormal{[} DISTINCT \textnormal{]} $a_1$ \textnormal{[} MIN \textnormal{|} MAX \textnormal{|} DIFF \textnormal{]}, ..., $a_m$ \textnormal{[} MIN \textnormal{|} MAX \textnormal{|} DIFF \textnormal{]} \\
ORDER BY ...
\end{sql}

We extended the above syntax a bit further, so we additionally provide syntax to specify
\begin{itemize}
\item treatment of NULL values (\inlinesql{NULLS FIRST} and \inlinesql{NULLS LAST})
\item usage of user defined order on attribute (\inlinesql{USING \emph{Op}}) and
\item operational aspects of skyline computation, such as \emph{algorithm} (BNL, SFS, \ldots), \emph{tuple window size and policy}, and more.
\end{itemize}

To describe the syntax we use 
\subsection{Railroad Diagrams}

\railalias{selectclause}{select\_clause}
\railalias{targetlist}{target\_list}
\railalias{intoclause}{into\_clause}
\railalias{fromclause}{from\_clause}
\railalias{whereclause}{where\_clause}
\railalias{groupclause}{group\_clause}
\railalias{havingclause}{having\_clause}
\railalias{skylineclause}{skyline\_clause}
\railalias{sortclause}{sort\_clause}
\railalias{qualOp}{qual\_Op}
\railalias{cexpr}{c\_expr}

% \railparam{\thinline}
% \railparam{\thicklines}
\railtermfont{\ttfamily\upshape\tiny}
\railboxheight 12pt
\railinit

\begin{rail}

selectclause : 'SELECT' 'DISTINCT'? targetlist intoclause ? fromclause \\ whereclause ? ( groupclause havingclause ? ) ? \\ skylineclause ? sortclause ? ';';
skylineclause : 'SKYLINE OF' 'DISTINCT' ? ( skylineexpr + ',' ) skylineoptions ?;
skylineexpr: cexpr ( 'MIN' | 'MAX' | 'DIFF' | 'USING' qualOp ) (() | 'NULLS FIRST' | 'NULLS LAST') ;
skylineoptions : 'WITH' ('EF' (('EFSLOTS' '=' slots) | (('EFWINDOW' | 'EFWINDOWSIZE') '=' windowsize ) ) ?) ? \\ ( (('BNL' | 'SFS') windowoptions ?) | 'PRESORT') ? ;
windowoptions : ( 'SLOTS' '=' slots ) | (('WINDOW' | 'WINDOWSIZE') '=' windowsize ) ;

\end{rail}


In fact we defined the grammar a little bit different\footnote{see \srcref{src/backend/parser/gram.y} for details} but only in the aspect of the assozitivity of the entire \inlinesql{SKLYINE OF} clause, this is because \fixme{SQL92}{ref:SQL92} requires the following statement:
\begin{sql}SELECT foo UNION SELECT bar ORDER BY baz\end{sql}
to be parsed as 
\begin{sql}(SELECT foo UNION SELECT bar) ORDER BY baz\end{sql}
and not as
\begin{sql}SELECT foo UNION (SELECT bar ORDER BY baz)\end{sql}

\noindent{}For the \inlinesql{SKYLINE OF} clause we decided that the it should be left
assoziative, i.e. 
\begin{sql}SELECT * FROM foo UNION SELECT * FROM bar SKYLINE OF baz\end{sql}
will be parsed as 
\begin{sql}SELECT * FROM foo UNION (SELECT * FROM bar SKYLINE OF baz)\end{sql}

\noindent{}We did so, because we believe the \inlinesql{SKYLINE OF} clause is closer
related to the \inlinesql{GROUP BY} clause than to the \inlinesql{ORDER BY}
clause, therefore we parse it in the same way.

\subsection{Reserved Keywords}


\begin{verbatim}
> CREATE DATABASE SKYLINE;

ERROR:  syntax error at or near "SKYLINE"
LINE 1: CREATE DATABASE SKYLINE;
                        ^
\end{verbatim}
The solution is easy and the same for every reserved word, just quote it:
\begin{verbatim}
> CREATE DATABASE "SKYLINE";
\end{verbatim}

\section{Semantics}

\subsection{One dimensional \inlinesql{SKYLINE OF} is different to SQL \inlinesql{MIN}/\inlinesql{MAX}}
skyline 1d min neq sql min:

sql min, max, distinct != skyline min, max, distinct

wenn mehr als ein column in der result list ist

\subsection{Skyline in the presents of NULL values}
\begin{verbatim}
skyline by nulls first / nulls last

Hannes=# select * from foo skyline by (price) nulls first;

 id | price | dist
----+-------+------
  5 |       |    3

Hannes=# select * from foo skyline by distinct (price) max;

 id | price | dist
----+-------+------
  5 |       |    3
(1 row)

Hannes=# select * from foo skyline by distinct (price) max nulls last;

 id | price | dist
----+-------+------
  1 |    10 |    1

should in case of max nulls last be the default? on sorting it's not.
\end{verbatim}

\subsection{\inlinesql{SKYLINE OF} and \inlinesql{GROUP BY}}
\begin{verbatim}
skyline and group by

select year, count(*)
from logfile
group by year
skyline by month min;


select year, count(*)
from logfile
group by year
skyline by year min;

==>

select year, count(*), month
from logfile
group by year, month
skyline by month min

hm? i.e. the skyline expr has to be added to the group by.
the result might be a bit unexpected if the skyline expr
is not part of the result nor of the group by clause


01 2000
02 2000
01 2001
01 2001

select year, count(*) from logfile group by year

2000 2
2001 2


select year, count(*), month from logfile group by year, month


2000 1 01
2000 1 02
2001 2 01

skyline by month min

2000 1 01
2001 2 01



skyline
  aggrate
     scan


Eder=# select id, count(*) from e2d100 group by id skyline by (d1) min;
ERROR:  column "e2d100.d1" must appear in the GROUP BY clause or be used in an aggregate function
\end{verbatim}

\subsection{\inlinesql{SKYLINE OF} is different to SQL:2003 \inlinesql{WINDOW} clause}

\todo{}{explain more}

\section{Pseudo Code}

\section{Sampling}

sampling --> min / max  ---> (0,1) --> ln   --> sum

\section{Cardinality and Cost Estimation}

\section{Usage of indizes}
skyline 1dim with index: in case of 1 dim skyline and the presence of an usable index, the fact that the index scan could be cancled early is not explored.


\section{Development}
\subsection{Debugging PostgreSQL}
Turing development it was very helpful to run the PostgreSQL backend in \texttt{--single} mode. A single backend, enter the query directly, edit & continue in msvc.


\chapter{Results\revision}
\label{chap:Results}

\section{Experiments}

\subsection{Random Dataset Generator}

N(0,1) in kossmann generator.txt

http://de.wikipedia.org/wiki/Normalverteilung\#Box-Muller-Methode

die zw�lfer regel scheint problematisch,

ein mal spektrum analysieren

ev. erf verwenden \#include <math.h>

\subsection{window size}
assumtions on windowsize:

the assumtion that at no time more than output\_tuples will be in the window can not be hold.

larger window and and less passes do not always mean better runtime

infact there is a break even point


\chapter*{Summary\revision}
\addcontentsline{toc}{chapter}{Summary\revision}



