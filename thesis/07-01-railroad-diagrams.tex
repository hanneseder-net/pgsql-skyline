%%%
%%% Railroad Diagrams.
%%%

\svnInfo $Id$

\section{SQL Extension}

In this section we explain the syntax and the semantics of the SQL
extension we implemened in order to support the skyline operator. The
extended query syntax resembles the syntax proposed in
\citep{Borzsonyi2001}.

To describe the syntax we use 
\subsection{Railroad Diagrams}

\railalias{selectclause}{select\_clause}
\railalias{targetlist}{target\_list}
\railalias{intoclause}{into\_clause}
\railalias{fromclause}{from\_clause}
\railalias{whereclause}{where\_clause}
\railalias{groupclause}{group\_clause}
\railalias{havingclause}{having\_clause}
\railalias{skylineclause}{skyline\_clause}
\railalias{sortclause}{sort\_clause}
\railalias{qualOp}{qual\_Op}
\railalias{cexpr}{c\_expr}

% \railparam{\thinline}
% \railparam{\thicklines}
\railtermfont{\ttfamily\upshape\tiny}
\railboxheight 12pt
\railinit

\begin{rail}

selectclause : 'SELECT' 'DISTINCT'? targetlist intoclause ? fromclause \\ whereclause ? ( groupclause havingclause ? ) ? \\ skylineclause ? sortclause ? ';';
skylineclause : 'SKYLINE' ( 'BY' | 'ON' ) 'DISTINCT' ? ( skylineexpr + ',' ) skylineoptions ?;
skylineexpr: cexpr ( 'MIN' | 'MAX' | 'DIFF' | 'USING' qualOp ) (() | 'NULLS FIRST' | 'NULLS LAST') ;
skylineoptions : 'WITH' ('EF' (('EFSLOTS' '=' slots) | (('EFWINDOW' | 'EFWINDOWSIZE') '=' windowsize ) ) ?) ? \\ ( (('BNL' | 'SFS') windowoptions ?) | 'PRESORT') ? ;
windowoptions : ( 'SLOTS' '=' slots ) | (('WINDOW' | 'WINDOWSIZE') '=' windowsize ) ;

\end{rail}


\newcommand\inlinesql[1]{{\tt #1}}
\newcommand\sql[1]{\inlinesql{#1}}
\newcommand\fixme[1]{{\bf FIXME (#1)}}
\newcommand\srcref[1]{{\tt #1}}

In fact the grammar is a defined a bit differnt, this is because SQL92
\fixme{ref:SQL92} requries
\sql{SELECT foo UNION SELECT bar ORDER BY baz}
to be parsed as 
\sql{(SELECT foo UNION SELECT bar) ORDER BY)}
and not as
\sql{SELECT foo UNION (SELECT bar ORDER BY baz)}
see \srcref{src/backend/parser/gram.y} for details.

For the \inlinesql{SKYLINE} clause we decided that the it should be left
assoziative, i.e. \\
\sql{SELECT * FROM foo UNION SELECT * FROM bar SKYLINE BY baz} \\
to be parsed as \\
\sql{SELECT * FROM foo UNION (SELECT * FROM bar SKYLINE BY baz)}. 

We did so because we believe the \inlinesql{SKYLINE} clause is closer
related to the \inlinesql{GROUP BY} clause than to the \inlinesql{ORDER BY}
clause, therefore we parse it in the same way.

